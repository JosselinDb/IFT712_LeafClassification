\subsection{Perceptron Multi-Couches (Multi Layer Perceptron)}

On implémente aussi un réseau neuronal simple, le \href{https://scikit-learn.org/stable/modules/generated/sklearn.neural_network.MLPClassifier.html}{perceptron multi-couches}. Il consiste en plusieurs couches pleinement connectées de perceptrons, permettant de faire des transformations non linéaires de notre espace de points.\\

La méthode dispose de plusieurs hyperparamètres :

\noindent
\begin{tabularx}{\textwidth}{|X|X|}
    \hline
    Nombre et taille des couches  & Le nombre de couches cachées et nombre de neurones par couches.\\\hline
    Fonction d'activation & La fonction d'activation des perceptrons, parmi sigmoïde, tanh et ReLU. \\\hline
    Solveur (et ses paramètres propres) & Le solveur utilisé pour la rétropropagation, parmi lbfgs, sgd (et son taux d'apprentissage et moment) et adam (et ses taux de décomposition des vecteurs de moments) . \\\hline
\end{tabularx}