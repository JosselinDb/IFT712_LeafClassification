\subsection{Forêt d'arbres décisionnels (Random Forest)}

On utilise ensuite la \href{https://scikit-learn.org/stable/modules/generated/sklearn.ensemble.RandomForestClassifier.html}{forêt d'arbres décisionnels}, une combinaison de modèles. Cette méthode consiste en la combinaison de plusieurs arbres décisionnels afin d'approcher au mieux la solution.\\

La méthode dispose de plusieurs hyperparamètres :

\noindent
\begin{tabularx}{\textwidth}{|X|X|}
    \hline
    Nombre d'arbres  & Le nombre d'arbres dans la forêt.\\\hline
    Critère de qualité & La fonction qui mesure la qualité d'une séparation, parmi l'impureté de Gini et l'entropie. \\\hline
\end{tabularx}