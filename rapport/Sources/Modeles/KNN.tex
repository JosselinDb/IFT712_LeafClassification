\subsection{K Plus Proches Voisins (K Nearest Neighbors)}

La première méthode qu'on a choisi d'utiliser est sûrement la plus classique : les \href{https://scikit-learn.org/stable/modules/generated/sklearn.neighbors.KNeighborsClassifier.html}{K plus proches voisins}. Cette méthode consiste à évaluer la classe d'une donnée en prenant la classe la plus représentée parmi ses $k$ "plus proches" voisins pour une mesure choisie.\\

La méthode dispose de plusieurs hyperparamètres :

\noindent
\begin{tabularx}{\textwidth}{|X|X|}
    \hline
    k & Le nombre de voisins à prendre en compte.\\\hline
    Algorithme (et leurs paramètres propres) & L'algorithme utilisé pour la recherche des plus proches voisins (parmi \href{https://scikit-learn.org/stable/modules/generated/sklearn.neighbors.BallTree.htm}{BallTree}, \href{https://scikit-learn.org/stable/modules/generated/sklearn.neighbors.KDTree.html}{KDTree}, et la force brute). \\\hline
    Métrique (et leurs paramètres propres) & La métrique utilisée pour mesurer la distance entre deux points. \\\hline
    Poids & Mesure d'importance des voisins en fonction de leur distance au point à classer.\\\hline
\end{tabularx}